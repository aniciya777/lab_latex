\documentclass[a4paper,14pt]{extreport}
\usepackage[T2A]{fontenc}
\usepackage[utf8x]{inputenc}
\usepackage[english,russian]{babel}
\usepackage{amssymb, amsmath}
\usepackage{theorem}
\usepackage{xcolor}
\usepackage{hyperref}
\definecolor{linkcolor}{HTML}{000000}
\definecolor{urlcolor}{HTML}{0645AD}
\hypersetup{pdfstartview=FitH,  linkcolor=linkcolor,urlcolor=urlcolor, colorlinks=true}


\begin{document}
	
	\begin{center}
		\Large
		\textbf{Линейная регрессия}
	\end{center}

	Линейная регрессия — метод восстановления зависимости между двумя переменными. Линейная означает, что мы предполагаем, что переменные выражаются через уравнение вида:
	
	\begin{equation}
		y = ax + b + \varepsilon.
		\label{o} 
	\end{equation}

	Эпсилон в формуле \eqref{o} — это ошибка модели. 
	
	\begin{center}
		\Large
		\textbf{Метод наименьших квадратов}
	\end{center}

	Суть МНК заключается в том, чтобы отыскать такие параметры $\Theta$, чтобы предсказанное значение было наиболее близким к реальному. Математически это выглядит так:
	
	\begin{equation}
		\sum\limits_{i=0}^n(y_i - \hat{y_i})^2 \longrightarrow \min\limits_{\Theta}.
		\label{m} 
	\end{equation}

	  Требуется найти такой вектор $\Theta$ \eqref{m}, при котором выражение:
	  
	  \begin{equation}
	  	\sum\limits_{i=0}^n(y_i - f(x_i, \Theta))^2,
	  \end{equation}
	  достигает минимума.
	  
	  Длинные преобразования из статьи. Демонстрация многострочной формулы:
	
	  \begin{multline}
	  	(Y-A\Theta)^T(Y-A\Theta) =\\
	  	= Y^TY-(A\Theta)^TY-Y^TA\Theta+(A\Theta)^TA\Theta.
	  \end{multline}
	  
	  А вот ссылка на статью \href{https://habr.com/ru/post/307004/}{https://habr.com/ru/post/307004/}.
	  
\end{document}